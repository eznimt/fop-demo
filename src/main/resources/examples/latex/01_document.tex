\documentclass[a4paper, 12pt]{scrartcl}
% globale Variablen
\newcommand{\doctitle}{Übungen zur Programmierung}
\newcommand{\doctitleshort}{Programmierung}
\newcommand{\theauthor}{Tobias Mintzlaff}
\newcommand{\email}{}
% Kopf- und Fusszeile
\usepackage{pdfpages, fancyhdr}
\pagestyle{fancy}
\fancyhf{}
% Kopf links
\fancyhead[L]{\doctitleshort}
% Kopf rechts
\fancyhead[R]{\nouppercase{\leftmark}}
\renewcommand{\headrulewidth}{0,4pt}
% Fuss links
\fancyfoot[L]{\theauthor, 2026}
% Fuss rechts
\fancyfoot[R]{Seite \thepage}
\renewcommand{\footrulewidth}{0,4pt}
\addtolength{\headwidth}{-8mm}
% schlampiger Absatz
\sloppy
\frenchspacing
% Einruecken 1. Zeile des Absatz
\setlength{\parindent}{0em}
% Sonderzeichen
%\usepackage[latin1]{inputenc} % Windows
\usepackage[utf8]{inputenc} % Linux
% Silbentrennung
\usepackage[ngerman]{babel}
% Umlaute ersetzen
\usepackage[T1]{fontenc}
% Seitenr�nder
\usepackage[a4paper, left=3cm, right=3cm, top=3cm, bottom=3cm]{geometry}
% Dokumenteigenschaften
\usepackage[colorlinks=false,
	pdfborder={0 0 0},
	pdftitle={\doctitle},
	pdfauthor={\theauthor},
 	pdfsubject={doctitleshort},
 	pdfcreator={LaTeX},
 	pdfproducer={Minze's TeX-Template 2.2}]{hyperref}
\usepackage[official]{eurosym}
\usepackage{pifont}
\usepackage{comment}
\newcommand{\zb}{z.\,B.~}
\newcommand{\idr}{i.\,d.\,R.~}
\newcommand{\e}{\,\euro{}}
\newcommand{\qm}{\,m$^2$}
\newcommand{\dahe}{d.\,h.~}
\usepackage{graphicx}
\usepackage{amsmath, longtable}
\usepackage{listings}
\usepackage{color, framed}
\definecolor{codegreen}{rgb}{0,0.6,0}
\definecolor{codegray}{rgb}{0.5,0.5,0.5}
\definecolor{codepurple}{rgb}{0.58,0,0.82}
\definecolor{backcolour}{rgb}{0.95,0.95,0.92}

\lstdefinestyle{mystyle}{
	backgroundcolor=\color{backcolour},
	commentstyle=\color{codegreen},
	keywordstyle=\color{magenta},
	numberstyle=\tiny\color{codegray},
	stringstyle=\color{codepurple},
	basicstyle=\ttfamily\footnotesize,
	breakatwhitespace=false,
	breaklines=true,
	captionpos=b,
	keepspaces=true,
	numbers=left,
	numbersep=5pt,
	showspaces=false,
	showstringspaces=false,
	showtabs=false,
	tabsize=2
}

\lstset{style=mystyle}
\begin{document}
% ----- Verzeichnisse ----------------------------------------------------
\newpage
\pagenumbering{roman}
\tableofcontents
\newpage
\pagenumbering{arabic}
\section{Grundlagen}
\subsection{Datentypen und Variablen}
Bei Variablen handelt es sich um Speicherbereiche, in denen Werte gespeichert werden können, und der Datentyp gibt an, welche Werte erlaubt sind, \zb nur Ganzzahlen.
In einem Programm werden Daten verarbeitet, die sich in ihrer Art unterscheiden, \zb Zeichen, Zahlen oder logische Daten.

Es gibt acht primitive Datentypen.
\begin{itemize}
	\item[-] ganzzahlige Typen: byte, short, int, long
	\item[-] Gleitpunkttypen: float, double
	\item[-] char
	\item[-] boolean
\end{itemize}

\subsection{Wertebereiche}
Warum gibt es mehrere numerische Datentypen?
Welchen Datentyp kann man für die Zahl 96 verwenden?
\subsection{Arrays}
Ein Array ist ein Container\footnote{Das ist eine Fußnote}, der eine feste Anzahl von Werten eines einzelnen Datentyps enthält.
Der Datentyp wird bei der Deklaration festgelegt und kann nicht mehr verändert werden.
Ebenso wird die Länge festgelegt.
\begin{lstlisting}[language=Java]
	class Main {
    public static void main(String[] args) {
        // Array deklarieren mit dem Datentyp Integer
        int[] myArray;

        //Array besitzt einen Speicher fuer 4 Integer
        myArray = new int[4];

        // Array initialisieren mit den Werten: 33, 91, 27, 5
        myArray[0] = 33;
        myArray[1] = 91;
        myArray[2] = 27;
        myArray[3] = 5;

        // Zugriff auf alle Werte des Arrays bzw. Ausgabe der Werte
        System.out.println(myArray[0]);
        System.out.println(myArray[1]);
        System.out.println(myArray[2]);
        System.out.println(myArray[3]);
    }
}
\end{lstlisting}
\end{document}